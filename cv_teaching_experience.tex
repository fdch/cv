\title{ Course Assistantship }
\employer{ Computer Music Techniques}
\location{ FAS NYU}
\dates{ Fall 2016 }
\begin{position}
This course introduced students to the general theory of digital sound signals, and the techniques to synthesize and transform them. Some topics included: sampling theorem and sine waves, samples, reading/writing arrays, additive synthesis, filters and subtractive synthesis, among others. I gave lectures on video and graphics in GEM, graded homeworks, and assisted students throughout the course, but mostly during their final projects.
\end{position}

\title{ Professorship }
\employer{ Harmony and Counterpoint I}
\dates{ Fall 2015 }
\begin{position} 
Focused on practical, technical and collaborative work, this course introduced students to polyphonic music thinking and writing, from a composer's perspective. I designed the syllabus, gave lectures, graded, and worked with a Teaching Assistant who was in charge recitation lessons. 
\end{position}
% \employer{ New York University}
% \location{ Faculty of Arts and Science}


\title{ Teaching Assistantship }
\employer{ Harmony and Counterpoint II}
\dates{ Spring 2015 }
\begin{position}
This course situated the practice of harmony and counterpoint in a historical
context and examined the limits of tonality through a detailed study of chromatic harmony. I was in charge of recitation lessons and graded homework and practical excercises. 
\end{position}
% \employer{ New York University}
% \location{ Faculty of Arts and Science}

\title{ Teaching Assistantship }
\dates { Fall 2014 }
\employer{ Harmony and Counterpoint I}
\begin{position}
This course provided basic tools (conceptual and aural) for understanding diatonic (tonal relations within a single key) procedures. It focused primarily on harmony and voice-leading practices in tonal European art music of the common practice period of the 18 th century. I was in charge of recitation lessons and graded homework and practical excercises.
\end{position}
% \employer{ New York University}
% \location{ Faculty of Arts and Science}
% \dates{ 2010-2014 }
% \title{ Teaching Assistantship}
%  \begin{position}
% % { Composición Musical I}

% \end{position}
\title{ Teaching Assistantship}
\employer{ Composición Musical I}
\location{ Facultad de Artes - UNC}
\dates{ 2010-2014 }
\begin{position}
This course provided the basic tools and theory for music composition in the 21st century. My role as assistant in this course was to give several lectures and provide and grade excercises for students.
\end{position}
% \employer{ Universidad Nacional de Córdoba}
% \location{ Facultad de Artes}
%  \dates{ 2009 }
%  \title{ Teaching Assistantship}
%  \begin{position}
% { Audioperceptiva I}

% \end{position}
% \title{ Teaching Assistantship}
\employer{ Audioperceptiva I}
% \location{ Escuela de Artes - FFyH - UNC}
\dates{ 2007-2009 }
\begin{position}
This course helped music students from a very broad range of fields to develop musical skills such as aural perception, solfege, sight reading, polyrhythmic reading, harmonic dication, among others. My role as assistant in this course was to give several lectures and provide and grade excercises for students.
\end{position}
% \employer{ Universidad Nacional de Córdoba}
% \location{ Escuela de Artes - FFyH}
% \dates{ 2007 }
