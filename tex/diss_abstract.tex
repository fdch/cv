This dissertation outlines a framework to discuss the aesthetic agency of the database in music composition. I place my dissertation in relation to existing scholarship, artists, and developers working in the fields of music composition, computer science, affect, and ontology, with emphasis on the ubiquity of databases and on the need to reflect on their practice, particularly in relation to databasing and music composition. There is a database everywhere, anytime, always already affecting our lives; it is an agent in our aesthetic and political lives just as much as we are agents in its composition and performance. Database music lives in between computers and sound. My argument is that in order to conceptualize the agency of the database in music composition, we need to trace the history of the practice, in both its technical and its artistic use, so as to find nodes of action that have an effect on the resulting aesthetics.
