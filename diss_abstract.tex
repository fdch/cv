The constantly changing technological, social, and global context of today's ``data revolution'' has placed data at the center of media research. Accumulating in archives, or data-bases, data has allowed data-driven practices to emerge. In this dissertation, I understand, contextualize, and reveal music composition as a database practice. In resonance with the extensive literature on the intersection between music and computers, I situate this discussion within the question of the agency of technology in art. Because the existing literature on database practices is mainly visually oriented, I analyze this agency in terms of the relationship between image and sound, concluding that it is the use of the database what allows for new ways of thinking multimedia works of art. As an outcome of my dissertation, I propose an open-source library for multimedia composition combining computer vision and timbre analysis algorithms, in order to generate a joint database of image and audio descriptors suitable for live multimedia composition.
